\documentclass[en]{tuletter}
\usepackage{csquotes}
\signname{
  Jason K. Moore, PhD\\
  Assistant Professor, BioMechanical Engineering\\
  Delft University of Technology
}
\signimage{jkm-signature}

% Reference top left
\date{23 november 2025}
\ourref{NA}
\yourref{NA}
\contact{Jason K. Moore}
\phone{+31 010-844 6853}
\email{j.k.moore@tudelft.nl}
\subject{IATSS Review Response}

% Recipient left under reference
\toname{Editor}
\toaddress{International Association of\\Traffic and Safety Sciences}

% Sender right under logo
\fromname{Faculty of Mechanical Engineering\\BioMechanical Engineering}
\fromaddress{Mekelweg 2\\2628 CD Delft\\The Netherlands}
%\frompobox{}

\begin{document}
\makeheader
\opening{Dear Reviewers,}

Thank you for the careful review of our work. We have revised the paper to
address your comments both for clarification of various methods and details as
well as providing background and support for some of our choices. The revision
is an improved version based on your feedback.

\begin{displayquote}
  \textbf{Editor}: Thank you once more for your submission to IATSS Research.
  The reviewers have now completed their assessment of your manuscript and have
  recommended major revisions before the manuscript is being considered for
  publication. I recommend you read carefully the reviewers' suggestions and
  prepare a detailed response to be submitted along your revised manuscript. To
  facilitate the reviewer's work, I also recommend to highlight the lines in
  the new version of the manuscript where changes you introduce address the
  each reviewer's specific concerns.
\end{displayquote}

We address each concern below and have included a copy of the updated
manuscript with the changes highlighted for your convenience.

\begin{displayquote}
  \textbf{Reviewer \#1}: This paper describe attempt to reduce probability of
  falls by providing electro-mechanical assistance to bicycles, which generally
  become unstable at very low speeds. Using an electric bicycle with an assist
  motor attached to the steering axis as the test vehicle, we first conducted
  theoretical studies and confirmed that it can reach speeds of 2.9 km/h
  without a rider and 4.6 km/h with rigid rider. Furthermore, 26 young people
  participated in an experiment in which an external force was applied to the
  handlebars while the test vehicle was running on a treadmill, and it was
  confirmed that assistance reduced the fall probability.

  Indoor experiments under controlled conditions are extremely useful for
  studying vehicle steering stability. In particular, since it is difficult to
  conduct full-scale experiments on single-track vehicles, which lack static
  upright stability, the results of indoor experiments using a treadmill are
  extremely useful. By controlling the conditions, the analysis of variables
  has been precise, and interesting findings have been obtained.

  Please add some explanation on the following points:

  One of the main themes of this research is evaluating the effectiveness of
  automatic assist systems. For this evaluation, it is extremely important to
  understand the operating characteristics of the automatic assist system. The
  paper provides detailed descriptions of control models and theories (e.g.,
  root locus), but says very little about actual performance or operating
  characteristics. While the paper states that the maximum torque is 7 Nm as a
  specific performance feature, I believe that showing transient performance
  such as instantaneous maximum torque is also important in understanding
  response characteristics. Therefore, while theoretical explanations are
  important, there should also be data on things like the responsiveness of
  actual bicycles. While the paper provides a very detailed description of
  perturbations, it says very little about the specific performance of the
  assist system, which is the subject of the paper's evaluation. I would like
  the paper to provide more information on the assist system.
\end{displayquote}

The steer motor was developed by Bosch eBike Systems and is a proprietary, thus
we are not privy to the fine details of its operation. We have not performed
our own motor testing, as that is not generally our expertise or interest. But
we have now included extra vehicle performance tests in the form of weave mode
identification of the rider-less bicycle. Figure 2 has been reworked to show
the transient behavior of the steer motor controlled bicycle compared to the
theoretical model's predictions. For the two testing speeds we set the
controller gain factor to different values. We now show the gain factors needed
in the ideal bicycle dynamics model to cause matching motion of the actual
system. Note that these gain factors 3.9 and 5.2 are different than the prior
paper version of 8 and 10, respectively. The reason for this is that the model
does not include the motor dynamics (motor constant or winding resistance), nor
does it include the scrub torque effect of the tire on the ground. The
approximate factor of 2 difference in the gain factor accounts for these
un-modeled effects. The gain factor of 8 and 10 are the values we employ in the
microcontroller of the real system. We hope that this proves 1) that our model
can predict the steer controlled bicycle's motion and 2) gives an idea of the
performance of the controlled vehicle. We have adapted the text in section 2.2
to explain this.

\begin{displayquote}
  In this study, a fall is defined as both ``putting one's foot on the ground''
  and ``exceeds the width of treadmill'', but there is insufficient explanation
  for this. In 4.2, the relationship between ``exceeds the width of treadmill''
  and actual falls is discussed, but the explanation for ``putting one's foot
  on the ground'' is insufficient. ``Putting one's foot on the ground'' seem to
  occur with a much milder degree of instability than ``exceeds the width of
  treadmill''. In the actual experimental situation,``putting one's foot on the
  ground'' may have been extremely dangerous, equivalent to ``exceeds the width
  of treadmill'', but this is not clear from the manuscript. I would like this
  point to be described in the Methods or Discussion section. Furthermore, the
  occurrence of such dangerous situations is strongly related to the
  instructions given to subjects beforehand. I would like the Methods section
  to provide additional explanation about the content of the instructions, such
  as whether any explanation about risk avoidance was given or not.
\end{displayquote}

The two modes of falling are different. We have added more information in
section 2.4 to explain these two modes. If the participants were cycling on an
infinite plane and we perturb them to fall, then we would strictly use body
contact with the ground as the criteria to define a fall. We have now cited
research that indicates riders often attempt to remove their foot from the
pedal and place it on the ground to lessen or prevent a complete fall to the
ground, so using that as an indicator for defining a fall is justified. But we
have a limited width treadmill, so as speed increases a perturbation will cause
greater lateral deviations. If the wheel of the bicycle exits the width we had
to stop the treadmill for safety reasons. Neither of the fall modes on the
treadmill are ``extremely dangerous'' or ``dangerous'', based on our
experience, so we did not consider this aspect.

\begin{displayquote}
  According to the explanation of the experimental procedures, I understand
  that in this experiment, each force was tested only once, and the subjects
  were not allowed to fully familiarize themselves with the force. I believe
  that the interpretation of the results would be completely different if the
  number of trials with each force were small or large enough. The fact that
  the number of trials was analyzed as a significant variable in the regression
  also reflects this assumption. I would like you to add an explanation about
  this point in the experimental procedures or discussion section.
\end{displayquote}

The participants experienced 40 random force magnitudes after they experienced
all force magnitudes in the threshold determination stage. So some forces were
experienced more than once. We would have liked to remove the learning effect
by letting the participants train for a long duration, but this was not
practical. We choose to include a measure of learning in the statistics as an
alternative. We have now noted this in Section 2.4.

\begin{displayquote}
  The direction of the perturbation is said to be random from left to right,
  but I cannot decide whether it is appropriate to assume that it is truly
  random when analyzing the results. I believe that when the number of trials
  is small, the factors that affect the results are not only the magnitude of
  the force, but also the direction, which has a significant impact.
  Consideration and explanation is needed as to whether the results when a
  perturbation is in the opposite direction to the previous perturbation can be
  considered the same as the results when the perturbation is in the same
  direction. Furthermore, I presume that the results will be affected if the
  bicycle's posture at the time the perturbation is applied is leaning in the
  same direction as the perturbation, or if it is leaning in the opposite
  direction. If such conditions are taken into account in the analysis, please
  explain.
\end{displayquote}

We did consider this and normalized the time series measurements to the sign of
the perturbation as well as included the configuration at the time of
perturbation as independent variables. A variable could be included to indicate
if a perturbation was preceded by the same sign perturbation or not, but we
have chosen not to include this, as it was not in our original hypothesis. We
added text to clarify these things in Section 2.5.

\begin{displayquote}
  The age range of the subjects is listed in 2.4 Protocol, but I would like the
  average age to be included as well.
\end{displayquote}

We have included the average age, albeit that four participants' ages were
misplaced in the two years since we did this study, which is declared in a
footnote.

\begin{displayquote}
  \textbf{Reviewer \#2}: This paper investigates the effectiveness of an
  automatic balance-assist system designed to reduce falls at low speeds. The
  core hypothesis is that a bicycle stabilized by an automatically controlled
  steering motor will reduce the probability of a fall. To verify this
  hypothesis, a well-considered experiment was conducted, and the analysis of
  its results is clearly explained. However, there are some unclear points that
  require the authors' response. Please refer to the following comments and
  make the necessary revisions.

  Section 1: Introduction on page 3

  Regarding low-speed accidents, the authors state that ``single-vehicle
  accidents are associated with a surprisingly high proportion of reported
  serious injuries,'' citing examples that can cause loss of control, such as
  gusts of wind, collisions between handlebars, bags hanging from handlebars,
  or simply contact with road surface irregularities. However, this qualitative
  description alone does not fully explain how the mechanical perturbations
  applied to the handlebars in the laboratory, specifically vertical forces,
  quantitatively reproduce these events. Particularly, bicycles are thought to
  weave more at low speeds, meaning the handlebars are shaking. Therefore,
  additional literature review or supplemental explanation is required to
  demonstrate that this experiment is an effective and representative method
  for studying real-world low-speed accident scenarios.
\end{displayquote}

We have now included citations to papers that discuss the types and nature of
bicycle falls and expanded the text in Section 1 to better connect our study to
real life falls. We also explain the force superposition principle of dynamics
to argue that it matters little how the perturbation is delivered as long as we
force the system into an unrecoverable state. We did not apply vertical forces
to the handlebars, only forces in the longitudinal direction, maybe this was a
misunderstanding as it is stated in the paper and shown in Figure 3a.

\begin{displayquote}
  Section 2.4: Protocol on page 7

  Regarding the definition of a fall, additional explanation is needed for the
  authors' reasoning for treating the rider removing their foot from the pedal
  and placing it on the ground and the bicycle wheel exceeding the width of the
  treadmill belt as the same category in the analysis. Please explain whether
  the fall scenarios differ or are the same depending on bicycle speed or the
  method of applying disturbances. In other words, can scenarios where the
  rider places their foot on the ground and falls, and scenarios where the
  bicycle's wheel exceeds the treadmill width and falls, be considered the same
  fall process, though differing in degree?
\end{displayquote}

First, see the answer above to reviewer \#1's similar question as a partial
response to this question. The two fall modes are not technically the same and
do not simply differ in degree, but were necessary given the nature of the
experimental design. We did not record the category of the fall mode, so it is
not possible for us to investigate whether having a ternary factor instead of a
binary one would reveal any more insight. We added text in Section 2.4 to make
this clear. We recommend attention be paid to this in future similar studies.

\begin{displayquote}
  Section 2.6: Statistics on page 11

  The authors state that ``cluster-mean centering showed there to be no
  variation between participants,'' which allowed authors to use a simple
  single-level logistic regression model rather than a multilevel model. This
  assertion is a foundational premise for the authors' entire analysis.
  Therefore, the authors are required to present the statistical testing
  process for cluster mean centering.
\end{displayquote}

In Section 2.6, we have included the maximum intraclass correlation (ICC) of
the models that include the participant as a random effect, which is 3\%.
Including this random effect does lower the p-values by a small amount, since a
bit more variation is accounted for, but we decided to use the more
conservative approach and not include this random effect. The participants do
not have appreciable variation in how they respond to the perturbations after
the data is normalized per participant. This lack of variation was also
observed in the independent dataset collected in Marten Haitjema's MSc thesis.

\begin{displayquote}
  Section 4.4: Rider Skill and Experience on page 14

  This relates to the above point. While the authors state that ``Each
  individual participant had their own perturbation resistance threshold''
  regarding individual differences in the experiment, it is difficult for
  readers to understand how individual differences were accounted for in the
  regression analysis, beyond the evidence regarding resistance values in the
  experiment. It is necessary to present the basic statistics of the
  experimental results and provide a detailed explanation of how individual
  differences were considered in the analysis.
\end{displayquote}

We report the ICC of the multilevel model, as noted in the prior answer, to
show that the variation in participants was not an important factor. We also
were not interested in the differences among participants, we are only
interested in whether the system reduces the probability of anyone to fall.
The cluster mean centering removes the relative differences in the
participants' perturbation resistance threshold. We have clarified this in
Section 2.6.

\begin{displayquote}
  Figure 2 on page 6

  It is preferable to shorten the figure caption, move the legend into the
  figure itself, and move the detailed figure explanation into the main body
  text.
\end{displayquote}

We have shortened the figure caption and added a legend.

We hope that these changes satisfy your concerns and look forward to your
further feedback.

\closing{Sincerely,}
\end{document}
