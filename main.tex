\documentclass{article}

\usepackage{siunitx}

\title{Robotic Bicycle Balance Assist Reduces Probability of Falling When
Subjected to Mechanical Perturbations}

\begin{document}

\maketitle

\abstract{
  Bicycles are unstable at low speeds. We add a controlled steering motor to a
  consumer bicycle that stablizes the bicycle at low speeds and apply varying
  magnitude external handlebar perturbations to the bicycle while ridden on a
  treadmill. The probability of not recovering from a handlebar perturbation
  decreases when the balance assist is activated.
}

\section{Introduction}
%
Single actor bicycle crashes are associated with a large number of unreported
and reported injuries and even sometimes death. At low speeds (less than 15
km/h or so) bicycles are no longer self stable and become more difficult to
control for the rider. Low speed crashes may be reduced if the bicycle was self
stable. Since at least the 1980s it known that applying a steering torque
proportional to the roll rate can stabilize a single track vehicle at very low
speeds. If the robotic control of steering can stablize a bicycle, it may
reduce the control required by the rider to successfully manage balancing
tasks. We have developed such a balance assisting bicycle and desire to
evaluate whether it helps the rider in situations they are likley to fall.

\section{Methods}
%
\subsection{Bicycle}
%
We modified a standard electric bicycle (Gazelle XXX, Dieren, The Netherlands)
with a custom motor in the head tube cable of applying up to
7~\si{\newton\meter} of torque between the head tube and steer tube. A custom
motor controller converts the commanded amperage to applied torque. We measure
the speed of the rear wheel with an encoder (Magura XXX) and measure the roll
rate of the bicycle with a MEMs rate gyroscope. The balance assist control
algorithm is implemented on microprocessor (Teensy).

\subsection{Balance Assist Control}
We use a forward speed gain scheduled proportional roll rate positive feedback
controller to stabilize the bicycle. Scaling the proprtional feedback gain
linearly with respect to speed stabilizes the normally unstable weave mode of
the bicycle from speeds of X~\si{\kilo\meter\per\hour} to
X~\si{\kilo\meter\per\hour} as shown in Figure~\ref{fig:eigenvalues}.
%
\begin{figure}
  \label{fig:eigenvalues}
\end{figure}

\subsection{Perturbations}
%
We apply longitudinal forces to the ends of each handlebar using an adapted
Bump'Em system~\cite{todo} making use of four motors working in tandem. The
four motors are programmed to apply a light force at all times to keep the
ropes taught and to track a commanded force profile using a PID controller
running on a microprocessor (Arduino XXX, Italy). We measure the force applied
by each motor at the handlebar via four inline load cells with a max load of
XX~\si{\newton}. The commanded force profiles are designed to apply an external
pulsive torque to the front assembly (handlebars, forks, wheel) at magnitudes
varying from X to X. The four motors are arranged at the four corners of a 1
meter wide treadmill (make, Company, City) that can reach speeds of
18~\si{\kilo\meter\per\hour}. The general design is described in detail in the
Van Velde's MSc thesis~\cite{vanvelde2022}. Our modifications to that described
in the thesis include simplifying the controller with a inexpensive
microcontoller and the use of a simple safety harness.

\subsection{Protocol}
%
Subjects wore a helmet and fall safety harness attached to the ceiling. We
allowed them practive riding on the treadmill until they said they were
comfortable enough to apply perturbations. For most this was less than a
10~\si{\minute} warm up. We then ask the rider to ride for 90~\si{\second}
attempting to maintain the location of their front wheel on the center line of
the treadmill. We define a "fall" on the treadmill by two criteria: the rider
removes their foot from the pedal and places it on the ground and the bicycle
wheel exceeds the width of the treadmill. We then applied perturbations in
random directions (clockwise or counter clockwise) increasing the magnitude by
X~\si{\newton\meter} each time. We log the magnitude that causes the first fall
to characterize that subject's nominal fall threshold. Following this we apply
20 perturbations of random magnitudes to the cyclist while they ride at a
constant speed and record which perturbations cause a fall. We let the cyclist
rest and then perform another 20 perturbations. We randomize whether the
balance assist system is on or off during the first or second set of
perturbations.

\subsection{Subjects}
%
We recruited X healthy young adults to participate in the experiments. X
subjects performed the experiments at 6~\si{\kilo\meter\per\hour} and X
subjects performed the experiments at 10~\si{\kilo\meter\per\hour}. The
10~\si{\kilo\meter\per\hour} experiments occured X weeks before the
6~\si{\kilo\meter\per\hour}.

\subsection{Measurements}

\section{Results}

\section{Discussion}

\section{Conclusion}

\end{document}
