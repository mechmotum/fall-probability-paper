\documentclass{article}

\usepackage{amsmath}
\usepackage[margin=25mm]{geometry}
\usepackage{siunitx}  % use for all units
\usepackage{todonotes}

\title{Robotic Bicycle Balance Assistance Reduces Probability of Falling When
Subjected to Mechanical Perturbations}

\author{Marten T. Haitjema \and Leila Alizadehsaravi \and Jason K. Moore}

\begin{document}

\maketitle

\abstract{
  Bicycles are unstable at low speeds. We add a controlled steering motor to a
  consumer bicycle that stabilizes the bicycle at low speeds and apply varying
  magnitude external handlebar perturbations to the bicycle while ridden on a
  treadmill. The probability of not recovering from a handlebar perturbation
  decreases when the balance assist is activated.
}

\section{Introduction}
%
Single actor bicycle crashes are associated with a large number of unreported
and reported injuries and even sometimes death. At low speeds (less than 15
km/h or so) bicycles are no longer self stable and become more difficult to
control for the rider. Low speed crashes may be reduced if the bicycle was self
stable. Since at least the 1980s it known that applying a steering torque
proportional to the roll rate can stabilize a single track vehicle at very low
speeds. If the robotic control of steering can stabilize a bicycle, it may
reduce the control required by the rider to successfully manage balancing
tasks. We have developed such a balance assisting bicycle and desire to
evaluate whether it helps the rider in situations they are likely to fall.

\section{Methods}
%
\subsection{Bicycle}
%
We modified a standard electric bicycle (Gazelle XXX, Dieren, The Netherlands)
with a custom motor in the head tube cable of applying up to
7~\si{\newton\meter} of torque between the head tube and steer tube. A custom
motor controller converts the commanded amperage to applied torque. We measure
the speed of the rear wheel with an encoder (Magura XXX) and measure the roll
rate of the bicycle with a MEMs rate gyroscope. The balance assist control
algorithm is implemented on microprocessor (Teensy).

\subsection{Balance Assist Control}
We use a forward speed \(v\) gain scheduled proportional roll rate positive
feedback controller to stabilize the bicycle. The commanded steer torque
\(T_\delta\) from the steer motor follows
\begin{align}
  T_\delta = k(4.7\si{\meter\per\second} - v)\dot{\phi}
\end{align}
where 4.7~\si{\meter\per\second} is the weave critical speed predicted from the
open loop bicycle rigid rider dynamics. We use \(k=8\) as the low gain value
and \(k=10\) as a high gain value. Scaling the proportional feedback gain
linearly with respect to speed stabilizes the normally unstable weave mode of
the bicycle from speeds of X~\si{\kilo\meter\per\hour} to
X~\si{\kilo\meter\per\hour} as shown in Figure~\ref{fig:eigenvalues}.

\todo[inline]{Show open loop bicycle rigid rider eigenvalue vs speed and then
the closed loop versions with the 8 and 10 gain settings. Three plots.}

\begin{figure}
  \label{fig:eigenvalues}
\end{figure}

\subsection{Perturbations}
%
We apply longitudinal forces to the ends of each handlebar using an adapted
Bump'Em system~\cite{Tan2020} making use of four motors working in tandem. The
four motors are programmed to apply a light force at all times to keep the
ropes taught and to track a commanded force profile using a PID controller
running on a microprocessor (Arduino XXX, Italy). We measure the force applied
by each motor at the handlebar via four inline load cells with a max load of
XX~\si{\newton}. The commanded force profiles are designed to apply an external
pulsive torque to the front assembly (handlebars, forks, wheel) at magnitudes
varying from X to X. The four motors are arranged at the four corners of a 1
meter wide treadmill (make, Company, City) that can reach speeds of
18~\si{\kilo\meter\per\hour}. The general design is described in detail in the
Van De Velde's MSc thesis~\cite{vandeVelde2022}. Our modifications to that
described in the thesis include simplifying the controller with a inexpensive
microcontroller and the use of a simple safety harness.

\subsection{Subjects}
%
We recruited 26 able-bodied young adults (20-36 years old) to participate in
the experiments. The subjects all were confident in their cycling skills and
had cycled at least once in the last month. Eleven subjects performed the
experiments at 1.7~\si{\meter\per\second} (6~\si{\kilo\meter\per\hour}) and
fifteen subjects performed the experiments at 2.8~\si{\meter\per\second}
(10~\si{\kilo\meter\per\hour}). All subjects consented to the experiment and
could decline to continue at any time.  The study was approved by Delft
University of Technology's Human Research Ethics Council (\#XXX).

\subsection{Two Experiments}
%
The 10~\si{\kilo\meter\per\hour} experiments occurred X weeks before the
6~\si{\kilo\meter\per\hour}.

\subsection{Protocol}
%
Subjects wore a helmet and fall safety harness attached to the ceiling. We
allowed them practice riding on the treadmill until they indicated they were
comfortable enough to apply perturbations. For most this was less than a
10~\si{\minute} warm up. We then ask the rider to ride for 90~\si{\second}
attempting to maintain the location of their front wheel on the center line of
the treadmill. We define a "fall" on the treadmill by two criteria: the rider
removes their foot from the pedal and places it on the ground and the bicycle
wheel exceeds the width of the treadmill. We then applied perturbations in
random directions (clockwise or counter clockwise) increasing the magnitude by
X~\si{\newton\meter} each time. We log the magnitude that causes the first fall
to characterize that subject's nominal fall threshold. Following this we apply
20 perturbations of random magnitudes to the cyclist while they ride at a
constant speed and record which perturbations cause a fall. We let the cyclist
rest and then perform another 20 perturbations. We randomize whether the
balance assist system is on or off during the first or second set of
perturbations.

Following the initial threshold determination, we choose perturbation forces
according to a random and adaptive staircase procedure applying
perturbations above and below the initial perturbation threshold, while
allowing small progression of the perturbation threshold to accommodate
learning effects. The goal of this adaptive staircase procedure is to have
participants fall for approximately 50\% of the time. The procedure
consists of twenty perturbations per condition. Participants undergo two
conditions: twenty perturbations with the balance-assist turned on and twenty
perturbations with the balance-assist turned off (in randomized order).

Five possible perturbation forces are determined based on the initial
perturbation threshold estimate: the initial estimate itself, two perturbations
lower than the initial estimate and two perturbations higher than the initial
estimate.  The five possible perturbations are separated by 10 N steps. For
example, if the initial estimate of the perturbation threshold of a participant
is 80 N, the five possible perturbations are 60, 70, 80, 90 and 100 N. Which
one of these five perturbations is chosen is determined at random. If the
perturbation results in a fall, the estimate of the perturbation threshold is
decreased with 10 N, and vice versa if the perturbation did not result in a
fall. Five new possible perturbation forces are determined around the updated
perturbation threshold, and a new perturbation is selected at random. This
process iterates until twenty perturbations are applied.

\subsection{Measurements}
%
We measure the time histories of the Bump'Em delivered perturbation forces and
the bicycle's steer angle, roll angle, roll angular rate, and forward speed.
%
\begin{table}
  \caption{Raw measurements}
  \begin{tabular}{llll}
    Measure & Variable & Units & Sensor \\
    Perturbation Order Number & \(j\) & integer & NA\\
    Perturbation Force \(l\)eft/\(r\)ight,\(f\)ront/\(r\)ear & \(F_{lf},F_{rf},F_{lr},F_{rr}\) & \si{\newton} & Load cell\\
    Steer Angle & \(\delta\) & \si{\degree} & encoder \\
    Roll Angle & \(\phi\) &  \si{\degree} & Kalman estimate \\
    Roll Angular Rate & \(\dot{\phi}\) &  \si{\degree\per\second} & rate gyroscope \\
    Balance Assist Gain & \(k\) & TODO & NA \\
    Bicycle Speed & \(v\) & Meters per second & wheel encoder \\
  \end{tabular}
\end{table}

Based on the findings in \cite{Haitjema2023}, we calculate several variables.
We use the angular impulse \(L\) due to the perturbation forces over a
0.3~\si{\second} duration to characterize the magnitude of delivered
perturbation. The duration is selected based on the duration of the commanded
force.

\begin{align}
  F_r = & F_{rf} - F_{rr} \\
  F_l = & F_{lf} - F_{lr} \\
  L = & \int_{0\si{\second}}^{0.3\si{\second}} l/2(F_r + F_l) dt
\end{align}

We characterize the skill of the rider by calculating the standard deviation of
the steer angle \(\sigma_\delta\) over the 90~\si{\second} period they were
asked to maintain their position at the center of the treadmill. At the
initiation of each perturbation we log the steer and roll angles to characterize
the configuration of the bicycle when perturbed. The gain setting on the
balance assist controller indicates if the assistance is off \(k=0\) or on at
two different levels: low \(k=8\) or high \(k=10\). A recovery from the
perturbation is successful if the person neither put their foot down onto the
treadmill surface or either wheel of the bicycle exits the width of the
treadmill belt. We record this as a binary variable \(f\).

\subsection{Statistics}
%
Our primary hypothesis is that the balance assistance controller will reduce
the probability of falling when perturbed externally at the handlebar. We have
a single binary outcome variable \(f\) that is dependent on several possible
explanatory independent variables including the binary balance assistance on or
off.

\begin{table}
  \begin{tabular}{lll}
    \(L\) & \si{\newton\meter\second} & angular impulse of the applied
    perturbation torque \\
    \(c\) & integer & order number of perturbation \\
    \(\sigma_\delta\) & standard deviation of steer angle during Phase X \\
    \(k\) & TODO & gain of balance assist control \\
    \(\delta_0\) & \si{\degree} & steer angle at start of perturbation \\
    \(\phi_0\) & \si{\degree} & roll angle at start of perturbation \\
    \(v\) & \si{\meter\per\second} & forward speed (6 or 10) \\
    \(f\) & boolean & outcome of perturbation: did not fall, did fall \\
  \end{tabular}
\end{table}

We evaluate this hypothesis using a multivariate logistic regression model.

\begin{align}
  f_{ij} | p_{ij} \sim \textrm{Bern}(p_{ij})
\end{align}

\(f_{ij}\) is the binary outcome of perturbation \(j\) of participant \(i\)
which follows a Bernoulli distriction given the probability \(p_{ij}\) that a
fall occurs. The log-odds of the probabiliy is then a linear function of our
independent variables.

\begin{align}
  \log\left(\frac{p_{ij}}{1 - p_{ij}}\right) = \Beta + \sum^K_{k=0} \alpha_k x^k_{ij}
\end{align}

\todo[inline]{we already use k for gain}

All of variables are scaled such they they have a mean of zero and a standard
deviation of one. We use cluster-mean centering with the clusters being an
individual subject.

We divide the analysis into two separate model evalautions, one for the
6~\si{\kilo\meter\per\hour} \(k=10\) trials and one for the
10~\si{\kilo\meter\per\hour} \(k=8\) trials and we evaluate our hypothesis for
each set of data.

\todo[inline]{Marten writes "The skill variable is not included, because it was
not collected correctly for all participants." This needs to be addressed.}

\section{Results}

\section{Discussion}

\section{Conclusion}

\bibliographystyle{plain}
% NOTE : this file is automatically generated from Zotero, do not edit
% manually!
\bibliography{references}

\end{document}
